\documentclass[a4paper]{article}

\usepackage[T1]{fontenc}
\usepackage[utf8]{inputenc}
\usepackage{polski}
\usepackage{graphicx}
\usepackage{hyperref}




\title{Programowanie Urządzeń Mobilnych \\ Laboratorium \\ \textbf{LISTA 5}}
\author{Rafał Lewandków}
\date{termin oddania: 07.02.2023}
\begin{document}
\maketitle
    

\section*{BankApp - 10 pkt}

Przygotuj aplikację wykorzystując \url{https://random-data-api.com/api/v2/banks} api. Aplikacja wyświetla listę 20 banków w \textbf{RecyclerView}. Aplikacja zawiera jeden fragment.

\begin{itemize}
\item \textbf{1 pkt} -- Wykorzystaj \textbf{Retrofit} oraz \textbf{LoggingInterceptor} do połączenia z api

\item \textbf{1 pkt} -- Wyświetl listę banków w \textbf{RecyclerView}

\item \textbf{2 pkt} -- Aplikację wykonaj w architekturze \textbf{MVVM}

\item \textbf{2 pkt} -- Wykorzystaj w aplikacji bibliotekę \textbf{Dagger-Hilt} do wykonania \textit{dependency injection} \textbf{OkHttp, Retrofit, Repository}

\item \textbf{3 pkt} -- Dodaj do aplikacji \textit{Offline Caching} z wykorzystaniem biblioteki \textbf{ROOM} - przy wykonaniu zapisu, do bazy powinny zostać dodane tylko elementy wcześniej nieobecne w wersji lokalnej.

\item \textbf{1 pkt} -- Wykorzystaj w aplikacji bibliotekę \textbf{Dagger-Hilt} do wykonania \textit{dependency injection} \textbf{Database}
\end{itemize}

\section*{OCENY}
Maxymalna liczba punktów: 10\\\
Oceny:\\
5.0 - 10 pkt\\
4.5 - 9 pkt\\
4,0 - 8 pkt\\
3,5 - 7 pkt\\
3,0 - 6 pkt
\end{document}

