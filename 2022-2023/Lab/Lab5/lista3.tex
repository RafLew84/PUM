\documentclass[a4paper]{article}

\usepackage[T1]{fontenc}
\usepackage[utf8]{inputenc}
\usepackage{polski}
\usepackage{graphicx}
\usepackage{hyperref}




\title{Programowanie Urządzeń Mobilnych \\ Laboratorium \\ \textbf{LISTA 3}}
\author{Rafał Lewandków}
\date{termin oddania: 09.12.2022}
\begin{document}
\maketitle
    

\section*{StudentHardLife - 10 pkt}

Przygotuj aplikację pozwalającą przechowywać (\textbf{SharePreferences, ROOM, SQLite}) i wyświetlać wszystkie listy zadań z przedmiotów. Aplikacja również umożliwia dodanie elementu, usunięcie elementu oraz edycję elementu. 

\begin{itemize}
\item \textbf{2 pkt} -- Wykorzystaj \textbf{Jetpack Navigation} do utworzenia nawigacji w aplikacji

\item \textbf{1 pkt} -- Aplikacja zawiera dwa fragmenty - wyświetlający wszystkie listy zadań, wyświetlający widok szczegółowy wybranej listy i umożliwiający edycję.

\item \textbf{1 pkt} -- Przygotuj model danych.

\item \textbf{2 pkt} -- Alikacja przechowuje dane w bazie danych (\textbf{ROOM, SQLite}) lub w pliku \textbf{SharedPreferences}

\item \textbf{3 pkt} -- Dodaj obsługę operacji dodawania, usunięcia i edycji wpisu

\item \textbf{1 pkt} -- Dodaj do aplikacji możliwość umieszczenia zdjęcia listy.
\end{itemize}

\section*{OCENY}
Maxymalna liczba punktów: 10\\\
Oceny:\\
5.0 - 10 pkt\\
4.5 - 9 pkt\\
4,0 - 8 pkt\\
3,5 - 7 pkt\\
3,0 - 6 pkt
\end{document}

